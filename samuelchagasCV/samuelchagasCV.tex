%!TEX TS-program = xelatex
%!TEX encoding = UTF-8 Unicode
% Awesome CV LaTeX Template for CV/Resume
%
% This template has been downloaded from:
% https://github.com/posquit0/Awesome-CV
%
% Author:
% Claud D. Park <posquit0.bj@gmail.com>
% http://www.posquit0.com
%
%
% Adapted to be an Rmarkdown template by Mitchell O'Hara-Wild
% 23 November 2018
%
% Template license:
% CC BY-SA 4.0 (https://creativecommons.org/licenses/by-sa/4.0/)
%
%-------------------------------------------------------------------------------
% CONFIGURATIONS
%-------------------------------------------------------------------------------
% A4 paper size by default, use 'letterpaper' for US letter
\documentclass[11pt,a4paper,]{awesome-cv}

% Configure page margins with geometry
\usepackage{geometry}
\geometry{left=1.4cm, top=.8cm, right=1.4cm, bottom=1.8cm, footskip=.5cm}


% Specify the location of the included fonts
\fontdir[fonts/]

% Color for highlights
% Awesome Colors: awesome-emerald, awesome-skyblue, awesome-red, awesome-pink, awesome-orange
%                 awesome-nephritis, awesome-concrete, awesome-darknight

\definecolor{awesome}{HTML}{2983f3}

% Colors for text
% Uncomment if you would like to specify your own color
% \definecolor{darktext}{HTML}{414141}
% \definecolor{text}{HTML}{333333}
% \definecolor{graytext}{HTML}{5D5D5D}
% \definecolor{lighttext}{HTML}{999999}

% Set false if you don't want to highlight section with awesome color
\setbool{acvSectionColorHighlight}{true}

% If you would like to change the social information separator from a pipe (|) to something else
\renewcommand{\acvHeaderSocialSep}{\quad\textbar\quad}

\def\endfirstpage{\newpage}

%-------------------------------------------------------------------------------
%	PERSONAL INFORMATION
%	Comment any of the lines below if they are not required
%-------------------------------------------------------------------------------
% Available options: circle|rectangle,edge/noedge,left/right

\name{Samuel Chagas de Assis}{}

\position{Programador - Biotecnologista}

\email{\href{mailto:samuel.achagas98@gmail.com}{\nolinkurl{samuel.achagas98@gmail.com}}}
\homepage{chagas98.github.io}
\linkedin{samuelchagass}

% \gitlab{gitlab-id}
% \stackoverflow{SO-id}{SO-name}
% \skype{skype-id}
% \reddit{reddit-id}


\usepackage{booktabs}

\providecommand{\tightlist}{%
	\setlength{\itemsep}{0pt}\setlength{\parskip}{0pt}}

%------------------------------------------------------------------------------



% Pandoc CSL macros
\newlength{\cslhangindent}
\setlength{\cslhangindent}{1.5em}
\newlength{\csllabelwidth}
\setlength{\csllabelwidth}{2em}
\newenvironment{CSLReferences}[3] % #1 hanging-ident, #2 entry spacing
 {% don't indent paragraphs
  \setlength{\parindent}{0pt}
  % turn on hanging indent if param 1 is 1
  \ifodd #1 \everypar{\setlength{\hangindent}{\cslhangindent}}\ignorespaces\fi
  % set entry spacing
  \ifnum #2 > 0
  \setlength{\parskip}{#2\baselineskip}
  \fi
 }%
 {}
\usepackage{calc}
\newcommand{\CSLBlock}[1]{#1\hfill\break}
\newcommand{\CSLLeftMargin}[1]{\parbox[t]{\csllabelwidth}{\honortitlestyle{#1}}}
\newcommand{\CSLRightInline}[1]{\parbox[t]{\linewidth - \csllabelwidth}{\honordatestyle{#1}}}
\newcommand{\CSLIndent}[1]{\hspace{\cslhangindent}#1}

\begin{document}

% Print the header with above personal informations
% Give optional argument to change alignment(C: center, L: left, R: right)
\makecvheader

% Print the footer with 3 arguments(<left>, <center>, <right>)
% Leave any of these blank if they are not needed
% 2019-02-14 Chris Umphlett - add flexibility to the document name in footer, rather than have it be static Curriculum Vitae


%-------------------------------------------------------------------------------
%	CV/RESUME CONTENT
%	Each section is imported separately, open each file in turn to modify content
%------------------------------------------------------------------------------



\begin{center}\rule{0.5\linewidth}{0.5pt}\end{center}

\begingroup
\fontsize{10}{12}\selectfont

\textit
{"Um biotecnologista em formação à aspirante em bioinformática, interessado em explorar a complexidade dos sistemas biológicos e a biodiversidade latinoamericana utilizando linhas de código e diálogo. Acredita na ciência aberta como um vetor de desenvolvimento científico e tecnológico. Possui experiência em R e Python para análises genéticas, modelagem de sistemas biológicos e visualização de dados. Atualmente, busca por uma posição que contribua para aprimorar seus conhecimentos em ciência de dados aplicado à biotecnologia"}
\endgroup

\hypertarget{formauxe7uxe3o}{%
\section{Formação}\label{formauxe7uxe3o}}

\begin{cventries}
    \cventry{Bacharelado em Biotecnologia}{Universidade Federal da Integração Latino-Americana (UNILA)}{Foz do Iguaçu, Paraná}{2017-Atualmente}{}\vspace{-4.0mm}
    \cventry{Técnico em Eletrotécnica integrado ao Ensino Médio}{Fundação Escola Técnica Liberato Salzano Vieira da Cunha}{Novo Hamburgo, Rio Grande do Sul}{2013-2018}{}\vspace{-4.0mm}
\end{cventries}

\hypertarget{experiuxeancias-acaduxeamicas}{%
\section{Experiências Acadêmicas}\label{experiuxeancias-acaduxeamicas}}

\hypertarget{extensuxe3o}{%
\subsection{Extensão}\label{extensuxe3o}}

\begin{cventries}
    \cventry{Monitor/\textit{Advisor}}{SynFronteras.Lab: Laboratório Virtual de Biologia Sintética}{Foz do Iguaçu, Paraná}{2022-Atualmente}{\begin{cvitems}
\item Ensino de Biologia sintética à estudantes de nível médio e superior
\item Coordenação de atividades de ensino
\end{cvitems}}
\end{cventries}

\hypertarget{pesquisa}{%
\subsection{Pesquisa}\label{pesquisa}}

\begin{cventries}
    \cventry{Bolsista de Iniciação Científica (UNILA/CNPq)}{Perfil alélico HLA-B de pacientes admitidos por COVID-19 na Unidade de Tratamento Intensivo (UTI).}{Foz do Iguaçu, Paraná}{2020-Atualmente}{\begin{cvitems}
\item Sequenciamento Sanger
\item Análise de dados
\item Visualização de dados \href{https://chagas98.github.io/hlafozpublic/SupplementaryMaterial.html}{(\underline{Código})}
\end{cvitems}}
\end{cventries}

\begin{cventries}
    \cventry{Coordenador \textit{DryLab}}{\textit{International Genetically Engineered Machine Competition} (iGEM)}{Foz do Iguaçu, Paraná}{2020-2021}{\begin{cvitems}
\item Fundador da \textit{UNILA_LatAm iGEM Team}
\item Modelagem Cinética e Análise de Sensibilidade Global \href{https://github.com/chagas98/iGEM2021_Modeling}{(\underline{Código})}
\item Predição de \textit{Toehold RNA Switches} \href{https://github.com/igemsoftware2021/CarpinchoToeholdDesigner}{(\underline{Código})}
\item Participação em Editais de Fomento
\end{cvitems}}
\end{cventries}

\begin{cventries}
    \cventry{Bolsista de Iniciação Científica (Fundação Araucária)}{Produção e Funcionalização de Nanocelulose Bacteriana}{Foz do Iguaçu, Paraná}{2018-2019}{\begin{cvitems}
\item Cultivo de \textit{Komagataeibacter xylinus}
\item Planejamento de Experimentos
\end{cvitems}}
\end{cventries}

\begin{cventries}
    \cventry{Trabalho de Conclusão Nível Médio-Técnico}{Sistema de Monitoramento da Contaminação do Ar pelo Herbicida Glifosato}{Novo Hamburgo, Rio Grande do Sul}{2015-2016}{\begin{cvitems}
\item Sensores eletroquímicos
\item Sistemas eletrônicos embarcados
\item Planejamento de Experimentos
\end{cvitems}}
\end{cventries}

\hypertarget{representauxe7uxe3o}{%
\subsection{Representação}\label{representauxe7uxe3o}}

\begin{cventries}
    \cventry{Representante Discente}{Colegiado do Curso de Biotecnologia - UNILA}{Foz do Iguaçu, Paraná}{2018-2019}{}\vspace{-4.0mm}
\end{cventries}

\begin{cventries}
    \cventry{Membro Fundador  CABiotec - UNILA }{Centro Acadêmico de Biotecnologia}{Foz do Iguaçu, Paraná}{2017-2019}{}\vspace{-4.0mm}
\end{cventries}

\hypertarget{experiuxeancias-profissionais}{%
\section{Experiências
Profissionais}\label{experiuxeancias-profissionais}}

\begin{cventries}
    \cventry{Estágio - Técnico em Eletrotécnica (720 horas)}{Parque Tecnológico Itaipu (PTI - Brasil)}{Foz do Iguaçu, Paraná}{2017-2018}{\begin{cvitems}
\item Auxiliar de Projetos Elétricos
\item Instrumentação e Eficiência Energética
\end{cvitems}}
\end{cventries}

\hypertarget{habilidades-e-qualificauxe7uxf5es}{%
\section{Habilidades e
Qualificações}\label{habilidades-e-qualificauxe7uxf5es}}

\begingroup
\fontsize{10}{12}\selectfont

\begin{itemize}
\tightlist
\item
  Linguagens de Programação: R, Python e Bash
\item
  \textit{Markup}: RMarkdown e HTML
\item
  Gerenciamento: Git/GitHub, GiHub Actions
\item
  Experiência em Sistemas UNIX/Linux
\item
  Sistemas eletrônicos embarcados (C++/Arduino)
\end{itemize}

\endgroup

\hypertarget{comunicauxe7uxe3o}{%
\section{Comunicação}\label{comunicauxe7uxe3o}}

\begin{cvhonors}
    \cvhonor{}{Nativo}{}{Português}
    \cvhonor{}{Intermediário à Avançado}{}{Inglês}
    \cvhonor{}{Intermediário}{}{Espanhol}
\end{cvhonors}

\hypertarget{pruxeamios}{%
\section{Prêmios}\label{pruxeamios}}

\begin{cvhonors}
    \cvhonor{}{Medalha de Ouro e Nomeação de Melhor Projeto de Nova Aplicação na \textit{International Genetically Engineering Machine Competition} (iGEM)}{}{2021}
    \cvhonor{}{Melhor Projeto de Escolha do Público no V Curso de Verão de Engenharia de Máquinas Biológicas - UFMG}{}{2020}
    \cvhonor{}{Melhor Apresentação no V Curso de Verão de Engenharia de Máquinas Biológicas - UFMG}{}{2020}
    \cvhonor{}{Primeiro Lugar. Menção Honrosa na área de Tecnologia e Produção no VII SEUNI - UNILA}{}{2019}
    \cvhonor{}{Menção Honrosa UNESCO pelo Projeto de Relevância Social na Conferência Internacional de Ciência e Tecnologia de Nível Médio - MOSTRATEC/UNESCO}{}{2016}
\end{cvhonors}


\label{LastPage}~
\end{document}
